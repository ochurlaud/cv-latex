\documentclass[skip,a4paper]{article}
\usepackage{ucs}
\usepackage[utf8x]{inputenc}
\usepackage[T1]{fontenc}
\usepackage{amsmath}
\usepackage{amsfonts}
\usepackage{amssymb}
\usepackage{color}
\usepackage{xcolor}
\usepackage{graphicx}
\usepackage{ulem} % option included for minimum surprise
\usepackage[francais]{babel}
\usepackage[top=1cm,bottom=1cm,right=1cm,left=1cm]{geometry}
\usepackage{titlesec}
\usepackage{hyperref}
\usepackage{url}
\usepackage{array}
\usepackage{marvosym}
\usepackage{tikz}

%-----Liens PDF-----%
\hypersetup{
	%backref=true, %permet d'ajouter des liens dans...
	%pagebackref=true,%...les bibliographies
	%hyperindex=true, %ajoute des liens dans les index.
	colorlinks=true, %colorise les liens
	breaklinks=true, %permet le retour à la ligne dans les liens trop longs
	urlcolor=black, %couleur des hyperliens
	linkcolor=black, %couleur des liens internes
	%bookmarks=true, %créé des signets pour Acrobat
	%bookmarksopen=true, %si les signets Acrobat sont créés,
	%les afficher complètement.
	pdftitle={CV Olivier CHURLAUD}, %informations apparaissant dans
	pdfauthor={Olivier CHURLAUD}, %dans les informations du document
	pdfsubject={CV Olivier CHURLAUD} %sous Acrobat.
}

\makeatletter

\pagestyle{empty}
\definecolor{redcv}{HTML}{C5000B}
\definecolor{graycv}{HTML}{666666}
\def\scale{0.9}
\def\descripscale{0.8}
\def\datescale{0.13}
\renewcommand{\arraystretch}{1.7}

\def\UrlFont{\em}

\titleformat{\section}{\LARGE\bfseries\color{redcv}}{\thesection}{1em}{}[{\titlerule[0.8pt]}]

\renewcommand\subsection{\@startsection{subsection}{2}{50px}%niveau toc / indentation
	{-3.5ex \@plus -1ex \@minus -.2ex}%
	{2.3ex \@plus.2ex}%
	{\reset@font\Large\bfseries}}

\newcommand{\itemcv}[2]{
		\textbf{\color{graycv}#1} & #2 \tabularnewline
	}
	
\newcolumntype{P}[1]{>{\raggedright}p{#1}}
\makeatother
\begin{document}
	\fontsize{8.5}{9.5}
	\selectfont

\begin{minipage}[c][4cm]{2.2cm}
	~\\~\\
	\begin{tikzpicture}
		\begin{scope}
			\clip [rounded corners=.5cm] (0,0) rectangle coordinate (centerpoint) (2.4,3cm); 
			\node [inner sep=0pt] at (centerpoint) {\includegraphics[width=2.4cm]{img/ID_ochurlaud}}; 
		\end{scope}
	\end{tikzpicture}
	\vfill
	~
\end{minipage}
\begin{minipage}[c][4cm]{5.5cm}
	~~~\textbf{Olivier CHURLAUD}
	\footnotesize
	\begin{itemize}
		\item[\bfseries @] \url{olivier@churlaud.com}
		\item[\bfseries \color{blue} in] {\scriptsize\url{ fr.linkedin.com/in/olivierchurlaud/}}
		\item[\Telefon]+33 (0)6 98 29 02 52
		\item[\Letter] 12, place du grand four \\
		79 370 MOUGON \\ 
		FRANCE
		\item[$\bullet$] né le 07 avril 1992
	\end{itemize}
\end{minipage}

\begin{minipage}[c][4cm]{10cm}
	\begin{minipage}[c]{7.10cm}
		\includegraphics[width=6.5cm]{img/ecl}
	\end{minipage}
	\hfill
	\begin{minipage}[c]{2.5cm}
		\includegraphics[width=2.1cm]{img/tuberlin}
	\end{minipage}
	
	\vfill
	
	\centering
	\LARGE
	\'Eleve Ingénieur\\
	Spécialité traitement du signal et de l'information
\end{minipage}

\hfill
\begin{minipage}{\scale\linewidth}
	\section*{\'Etudes}
	
	\begin{tabular}{p{\datescale\linewidth} P{\descripscale\linewidth}}
		\itemcv{2014 -- 2016}{TU Berlin : Master Elektrotechnik}
		\itemcv{2012 -- 2016}{École Centrale de Lyon : Préparation du diplôme d'ingénieur généraliste}
		\itemcv{2010 -- 2012}{Classe préparatoire Maths-Physiques (MP) au Lycée Pierre de Fermat de Toulouse (31)}
	\end{tabular}
	
	\section*{Expériences}
	\subsection*{Direction de projets}
	
	\begin{tabular}{p{\datescale\linewidth} P{\descripscale\linewidth}}
		\itemcv{2013 -- 2014}{
			\textbf{Projet de Recherche : Conception et création d'un module de transmission WPAN-WLAN} \\
			Collaboration avec le laboratoire INL (Lyon) \\
			\textit{Programmation C -- Conception KICAD -- Microcontrôleurs Microchip}
		}
		\itemcv{2012 -- 2013}{
			\textbf{Président de l'association ECLAIR} (association loi 1901 gérant l'informatique et le réseau de l'École Centrale de Lyon : 700 utilisateurs) : Mise en place d'un Cloud, Conception d'un méta-site interne pour les élèves et anciens élèves
		}
		\itemcv{~}{
			\textbf{Projet d’Étude : Création d'un outil de mesure des champs magnétiques de basse fréquence} \\
			Collaboration avec le Laboratoire Ampère (Lyon) \\
			\textit{Programmation MATLAB -- Mise en œuvre d'un magnétomètre}
		}
		\itemcv{2009 -- 2010}{
			\textbf{Projet de recherche en physique des particules en partenariat avec le CERN et l'IN2P3} \\
			Projet soutenu aux Olympiades de Physiques (1\ier prix), aux Faites de la Science (1\ier prix) \\
			\textit{Expérimentations -- Vulgarisation -- Conception d'un jeu de plateau sur le thème du LHC}
		}
		\itemcv{2008 -- 2009}{
			\textbf{Projet d'Atelier d'écriture en partenariat avec l'OuLiPo} (Ouvroir de Littérature Potentielle : groupement	d'écrivains qui travaille en se choisissant des contraintes de fond et de forme)
		}
	\end{tabular}
	
	\subsection*{Expériences professionnelles}
	
	\begin{tabular}{p{\datescale\linewidth} P{\descripscale\linewidth}}
		\itemcv{2014}{
			\textbf{Stage d'application : Télécom Bretagne, avec F.P. Andriulli} -- Analyse et développement de techniques et d'outils	expérimentaux et computationnels pour l'imagerie cérébrale par encéphalogramme
		}
		\itemcv{2013}{
			Stage ouvrier : \textbf{SNCF} - maintenance de locomotives BB26000 (Technicentre SNCF Oullins)
		}
		\itemcv{2010 et 2011}{
			Emploi saisonnier : manutention dans une entreprise de déménagement (BIARDEAU SARL, à Niort)
		}
	\end{tabular}
	
	\section*{Compétences}
	
	\begin{tabular}{p{\datescale\linewidth} P{\descripscale\linewidth}}
		\itemcv{Management}{Chef de projet, coordination d'équipes, lien entre administration et équipe}
		\itemcv{Électronique}{Programmation de microcontrôleurs Microchip, cartes électroniques de communication WPAN/WLAN}
		\itemcv{Informatique}{
			\textbf{Internet} HTML5, CSS, PHP, MySQL, Symfony2 -- \textbf{Administration système et réseaux} Linux, Bridging, VLAN, Virtualisation KVM, Apache2 -- \textbf{Programme} MS Access, MATLAB, C/C++, Python, \LaTeX
		}
		\itemcv{Langues}{
			\textbf{Anglais} (Lu, écrit, parlé : TOEFL 613pts) \\
			\textbf{Allemand} (Lu, écrit, parlé : Certificat de niveau B2 en allemand en 2014 {\small(Système de notation Européen)})
		}
	\end{tabular}
	
	\section*{Centres d’intérêt}
	
	\begin{tabular}{p{\datescale\linewidth} P{\descripscale\linewidth}}
		\itemcv{Culturel}{Littérature : littérature classique du XIX-XXe siècle et Nouveau Roman}
		\itemcv{~}{Musique, Cinéma : Cinéma d'auteur et des années 40-50}
		\itemcv{Sport}{Sports de montagne : ski, escalade, randonnée}
\end{tabular}
\end{minipage}

\end{document}