\usepackage{ucs}
\usepackage[utf8x]{inputenc}
\usepackage[T1]{fontenc}
\usepackage{amsmath}
\usepackage{amsfonts}
\usepackage{amssymb}
\usepackage[francais]{babel}
\usepackage[top=1cm,bottom=1cm,right=2cm,left=1cm]{geometry}
\usepackage{titlesec}
\usepackage{array}
\usepackage{enumitem}
%% URLs
\usepackage{hyperref}
\usepackage{url}
\usepackage{ulem} % em is underlined

%% Graphics and colors
\usepackage{tikz} % Round corner on picture
\usepackage{color} % enable color
\usepackage{xcolor} % Define own colors
\usepackage{graphicx} % Include picts.

%% Fonts
% cf http://www.tug.dk/FontCatalogue/seriffonts.html
\usepackage[default,osfigures,scale=0.95]{opensans}
% Symbols
\usepackage{marvosym}

\def\subject{cv}
\def\title{cv}
\def\author{author}

\definecolor{redcv}{HTML}{C5000B}
\definecolor{graycv}{HTML}{666666}
\def\scale{0.94}
\def\descripscale{0.8}
\def\datescale{0.13}
\renewcommand{\arraystretch}{1.7}

%-----Liens PDF-----%
\hypersetup{
	%backref=true, %permet d'ajouter des liens dans...
	%pagebackref=true,%...les bibliographies
	%hyperindex=true, %ajoute des liens dans les index.
	colorlinks=true, %colorise les liens
	breaklinks=true, %permet le retour à la ligne dans les liens trop longs
	urlcolor=black, %couleur des hyperliens
	linkcolor=black, %couleur des liens internes
	%bookmarks=true, %créé des signets pour Acrobat
	%bookmarksopen=true, %si les signets Acrobat sont créés,
	%les afficher complètement.
	pdftitle={\title}, %informations apparaissant dans
	pdfauthor={\author}, %dans les informations du document
	pdfsubject={\subject} %sous Acrobat.
}

\makeatletter

\pagestyle{empty}

\def\UrlFont{\em} % url are `em`, and `em` is underlined with ulem

\titleformat{\section}{\LARGE\bfseries\color{redcv}}{\thesection}{1em}{}[{\titlerule[0.8pt]}]
\titleformat{\subsection}{\normalfont\Large\bfseries}{\thesubsection}{1em}{}

\titlespacing*{\section}{0pt}{2ex plus 0.4ex minus .2ex}{1.3ex plus .2ex}
\titlespacing*{\subsection}{50pt}{2ex plus 0.4ex minus -0.2ex}{0.5ex plus .2ex}

\newcommand{\itemcv}[2]{\textbf{\color{graycv}#1} & #2 \tabularnewline}
	
\newcolumntype{P}[1]{>{\raggedright}p{#1}}
\makeatother